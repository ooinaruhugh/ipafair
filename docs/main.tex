\documentclass[oneside,12pt]{amsart}
\usepackage{macros}

\usepackage{standalone}
\usepackage{listings}
\lstset{
  basicstyle=\ttfamily,
  columns=fullflexible,
}

\usepackage[
bookmarksopen,
bookmarksnumbered,
%hidelinks,
pdftitle={Towards an incremental reentrant ASP-based solver for argumentation frameworks}
pdfauthor={Kamillo Hugh Ferry},
pdfkeywords={project,report},
pdfstartview={FitH},
]{hyperref}

\usepackage[backend=biber,style=numeric,backref,backrefstyle=none]{biblatex}
\addbibresource{literature.bib}

% Load LAST!
\usepackage[nameinlink]{cleveref}


\begin{document}
\title[Incremental AF solver]{Towards an incremental reentrant ASP-based solver for argumentation frameworks}
\author{Kamillo Hugh Ferry}
\address{}
\email{}

\begin{abstract}
    We devise a incremental ASP-based solving system for Dung's 
    abstract argumentation frameworks. This means 
    that we allow for dynamic changes to an input AF such as 
    adding or removing arguments and attacks while retaining
    the state of the underlying grounder and solver.
    
    This paper documents the design process and current state of 
    the incremental solver system. Also, we details some of
    the problems that were encountered and open issues of the 
    cosntructed system.
\end{abstract}

\maketitle

\section{Introduction}
The need for incremental algorithms arises when dealing with dynamically
changing inputs where it is desirable to process each such change relatively quickly
and where it is not desirable to process the input every time from scratch.
One example for such a setting are knowledge bases, and in particular,
abstract argumentation frameworks (AFs) that model a knowledge base.

Here, given an AF, we might have computed a set of possible extensions 
satisfying a certain semantic. But when we edit this AF, e.\,g.\ to react to 
a change in our knowledge base, and ask for the extensions of the resulting 
framework, we can try to keep information from previous solve iterations 
so we do not have to construct all extensions from scratch.

This approach to algorithmic problems trying to retain information between
the runs of an algorithm and incrementally solving a dynamically changing instance 
has been studied in literature before \cite{mans_incremental_2017,miltersen_complexity_1994,patnaik_dyn-fo_1997}.

An example taken from \citeauthor{patnaik_dyn-fo_1997}~\cite{patnaik_dyn-fo_1997} (Example 3.2), 
when asking for the parity of a string of $n$ bits, if we retain the
parity bit between calculations, we can actually express the answer using 
propositional formulas when the allowed edits are flipping bits. 

That is, suppose we have a data structure $S$ representing the state of a
string of $n$ bits called $S_x$, and the parity bit $b$ of this string.
We can define two operations $\mathsf{ins}(S,i)$ and $\mathsf{del}(S,i)$ 
that represent `setting the $i$-th bit to $0$ resp.\ $1$' that edit $S$ 
accordingly to the changes that were requested and those operations 
can be expressed by propositional formulae as in \cref{dynfo-parity}.

The main idea behind this example is that given a specific problem, we 
might be able to identify how a change to the input affects the solution 
and what kind of intermediate information has to be kept during each computation.

Applying this to abstract argumentation frameworks, we build upon the 
ASP-based argumentation system ASPARTIX~\cite{aspartix_website,aspartix} which already 
provides encodings for most common semantics of argumentation frameworks, albeit static ones.

ASPARTIX itself is implemented in the ASP system \emph{clingo}~\cite{clingo,potassco}, whose API 
for partial grounding and externally provided atoms is also heavily used 
to enable editing the input AF.

This is done by breaking apart the given encodings into parts that are affected
by the insertion of an argument and parts that are affected by the 
insertion of an attack. Whenever an argument or attack is to be added,
the respective partial encoding is applied. 

The detailed process of modifying the encodings and modifying 
the state of arguments and attacks is described in \cref{incr-semantics}.

Examples for the usage of the current system are given in \cref{usage}.
Current issues and possibilities for extension are described in \cref{issues,problems}.

The code for this project is included with this report and can be retrieved by
unzipping this file.

\begin{figure}
    \includestandalone[width=0.6\linewidth]{dynfo-parity}
    \caption{Operations $\mathsf{ins}$ and $\mathsf{del}$ for calculation of parity bits.}
    \label{dynfo-parity}
\end{figure}

\section{Usage}\label{usage}
Currently, our incremental system cannot be called from command-line. Instead, one has to 
instanciate a \texttt{IncrAFSolver} object in Python, as follows
\begin{mintedbox}{python}
from incraf import IncrAFSolver

af = IncrAFSolver(semantic, filename)\end{mintedbox}

To create an \texttt{IncrAFSolver} object, one only needs to provide 
a \texttt{semantic} since an AF can also be built up succesively.

Arguments and attacks can be added by the respective \texttt{add}
and \texttt{del} methods. Arguments can be named by strings
or integers

The following example adds to an framework \texttt{af}
two arguments named \texttt{2} and \texttt{b} and an attack between
them while removing an attack from arguments \texttt{1} to \texttt{a}.
\begin{mintedbox}{python}
af.add_argument(2)
af.add_argument("b")
af.add_attack(2,"b")
af.del_attack(1,"a")\end{mintedbox}

We can either enumerate all extensions satisfying the specified semantic with 
\texttt{solve\_enum}, or test for credulous or skeptical satisfiability 
with \texttt{solve\_cred} resp.\ \texttt{solve\_skept}. To get the accepted arguments
after calling one of the latter functions, one can use \texttt{extract\_witness}.

The following snippet describes the process of testing for skeptical satisfiability
and getting the skeptically accepted arguments.
\begin{mintedbox}{python}
af.solve_skept()

args = af.extract_witness()\end{mintedbox}

More complete examples are given in the files \texttt{example\_af1.jpynb} and \texttt{example\_af2.jpynb}

\section{Modification of the \textsc{Aspartix} encodings}\label{incr-semantics}
To allow us to accept changes to our input, we need to make two adjustments
to the way we feed the encodings to \emph{clingo}.
\begin{itemize}
    \item We need to add all atoms related to the input as \texttt{\#external} atoms.
    \item The encoding for the semantic needs to be split in two parts 
        \texttt{\#program add\_argument(v)} and \texttt{\#program add\_attack(v,u)}.
\end{itemize}

Declaring atoms as external instructs \emph{clingo} to omit \emph{certain simplifications}
\cite[3.1.15]{clingo_guide}. For example, external atoms are not removed from rules 
even if they do not appear in the head of any rule. 

Additionally, \emph{clingo} allows us to provide the truth value for any external atom. 
We use this to switch off arguments or attacks that have been deleted.

The need to break apart an encoding nito parts related to arguments and attacks 
comes from the fact that the initial grounding only introduces rules for 
arguments and attacks that existed at that point. 
Even if we add new atoms, the rules related to them that tie them to 
a possible solution do not exist.

We cannot re-ground simply since then we also try to redefine rules 
that already existed. Yet the process of grounding can be 
(overly simplified) thought of plugging in the actual values for variables occuring 
in rules. Thus, breaking apart the encoding allows us to only introduce 
only the necessary rules whenever adding an attack or argument.

For this, \emph{clingo} provides the \texttt{\#program} statement to 
denote program blocks accepting parameters. 

Take for example following encoding for naive semantics taken from ASPARTIX.
\begin{algobox}
\begin{lstlisting}
%% Guess a set S \subseteq A
in(X) :- not out(X), arg(X).
out(X) :- not in(X), arg(X).

%% S has to be conflict-free
:- in(X), in(Y), att(X,Y).

%% Check Maximality
okOut(X) :- in(Y), att(Y,X).
okOut(X) :- in(Y), att(X,Y).
okOut(X) :- att(X,X).
:- out(X), not okOut(X).
\end{lstlisting}
\end{algobox}

Obviously, the rules to guess a potential solution subset $S$ are tied 
to a specific argument, while the rules checking for conflict-freeness
are each tied to a specific attack. Also the rules checking for maximality 
depend on attacks.

This means that the guessing rules are added to the argument block
while the checking rules are added to the attack block.

Special care needs to be taken for the last rule. Since this one only references 
an argument, we need to actually add this rule to the argument block.
But since the \texttt{okOut(X)} atom does not occur in any rule head until 
there exists an attack involving \texttt{X}, we need to mark this atom as external.

The resulting incremental naive encoding then is as follows.
\begin{algobox}
\begin{lstlisting}
#program add_argument(v).
%% Guess a set S \subseteq A
in(v) :- not out(v),arg(v).
out(v) :- not in(v), arg(v).
#external okOut(v).
:-  arg(v), out(v), not okOut(v).

#program add_attack(v, u).
%% S has to be conflict-free
:- in(v), in(u), att(v,u).

%% Check Maximality
okOut(u) :- in(v), att(v,u).
okOut(v) :- in(u), att(v,u).
okOut(v) :- att(v,v).
\end{lstlisting}
\end{algobox}

With an incremental encoding as the one above, we can now add 
an argument or an attack by first adding an external atom,
if it did not exist before, and then performing the partial grounding 
of either \texttt{\#program add\_argument(v)} or \texttt{\#program add\_attack(v,u)}.

\section{Issues}\label{issues}
Currently, adding new attacks might not always work since \emph{clingo} 
throws an error that an atom gets redefined. This is related to the initial 
problem of redefining rules, but this phenomenon only occurs 
after calling a \texttt{solve} method once. 

Thus, the reason might be instead a conflict between a truth value 
that is already assigned to an atom and new rules causing this atom 
to change its assignment.


\section{Open Problems}\label{problems}
In additon to the blocking behaviour of \emph{clingo} described in \cref{issues} 
this solver does not provide a command-line interface yet.

Also, instead of trying to use \emph{clingo} with an incremental encoding, 
it is also possible to instead use a static encoding that produces a solution 
and possibly additional intermediate information and retain that. 
Then, when processing changes to an instance, we just modify the information 
that was retained and only call the solver from scratch when necessary.
This approach would be similar to the one presented by \citeauthor{mans_incremental_2017}~\cite{mans_incremental_2017}
for incremental computations.

\printbibliography

\end{document}