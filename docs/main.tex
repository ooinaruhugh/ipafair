\documentclass[oneside,12pt]{amsart}
\usepackage{macros}

\usepackage{standalone}

\usepackage[
bookmarksopen,
bookmarksnumbered,
%hidelinks,
pdftitle={Towards an incremental reentrant ASP-based solver for argumentation frameworks}
pdfauthor={Kamillo Hugh Ferry},
pdfkeywords={project,report},
pdfstartview={FitH},
]{hyperref}

\usepackage[backend=biber,style=numeric,backref,backrefstyle=none]{biblatex}
\addbibresource{literature.bib}

% Load LAST!
\usepackage[nameinlink]{cleveref}


\begin{document}
\title[Incremental AF solver]{Towards an incremental reentrant ASP-based solver for argumentation frameworks}
\author{Kamillo Hugh Ferry}
\address{}
\email{}

\begin{abstract}
    We devise a incremental ASP-based solving system for Dung's 
    abstract argumentation frameworks. This means 
    that we allow for dynamic changes to an input AF such as 
    adding or removing arguments and attacks while retaining
    the state of the underlying grounder and solver.
    
    This paper documents the design process and current state of 
    the incremental solver system. Also, we details some of
    the problems that were encountered and open issues of the 
    cosntructed system.
\end{abstract}

\maketitle

\section{Introduction}
The need for incremental algorithms arises when dealing with dynamically
changing inputs where it is desirable to process each such change relatively quickly
and where it is not desirable to process the input every time from scratch.
One example for such a setting are knowledge bases, and in particular,
abstract argumentation frameworks (AFs) that model a knowledge base.

Here, given an AF, we might have computed a set of possible extensions 
satisfying a certain semantic. But when we edit this AF, e.\,g.\ to react to 
a change in our knowledge base, and ask for the extensions of the resulting 
framework, we can try to keep information from previous solve iterations 
so we do not have to construct all extensions from scratch.

This approach to algorithmic problems trying to retain information between
the runs of an algorithm and incrementally solving a dynamically changing instance 
has been studied in literature before \cite{mans_incremental_2017,miltersen_complexity_1994,patnaik_dyn-fo_1997}.

An example taken from \citeauthor{patnaik_dyn-fo_1997}~\cite{patnaik_dyn-fo_1997} (Example 3.2), 
when asking for the parity of a string of $n$ bits, if we retain the
parity bit between calculations, we can actually express the answer using 
propositional formulas when the allowed edits are flipping bits. 

That is, suppose we have a data structure $S$ representing the state of a
string of $n$ bits called $S_x$, and the parity bit $b$ of this string.
We can define two operations $\mathsf{ins}(S,i)$ and $\mathsf{del}(S,i)$ 
that represent `setting the $i$-th bit to $0$ resp.\ $1$' that edit $S$ 
accordingly to the changes that were requested and those operations 
can be expressed by propositional formulae as in \cref{dynfo-parity}.

The main idea behind this example is that given a specific problem, we 
might be able to identify how a change to the input affects the solution 
and what kind of intermediate information has to be kept during each computation.

Applying this to abstract argumentation frameworks, we build upon the 
ASP-based argumentation system ASPARTIX \cite{aspartix} which already 
provides encodings for most common semantics of argumentation frameworks, albeit static ones.

ASPARTIX itself is implemented in the ASP system \emph{clingo}, whose API 
for partial grounding and externally provided atoms is also heavily used 
to enable editing the input AF.

This is done by breaking apart the given encodings into parts that are affected
by the insertion of an argument and parts that are affected by the 
insertion of an attack. Whenever an argument or attack is to be added,
the respective partial encoding is applied. 

The detailed process of modifying the encodings and 

\begin{figure}
    \includestandalone[width=0.6\linewidth]{dynfo-parity}
    \caption{Operations $\mathsf{ins}$ and $\mathsf{del}$ for calculation of parity bits}
    \label{dynfo-parity}
\end{figure}

\section{Usage}
\section{Modification of the \textsc{Aspartix} encodings}\label{incr-semantics}
\section{Issues}
\section{Open Problems}

\printbibliography

\end{document}